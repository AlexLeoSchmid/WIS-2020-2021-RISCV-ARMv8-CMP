\documentclass[conference]{IEEEtran}
\IEEEoverridecommandlockouts
% The preceding line is only needed to identify funding in the first footnote. If that is unneeded, please comment it out.
\usepackage{cite}
\usepackage{amsmath,amssymb,amsfonts}
\usepackage{algorithmic}
\usepackage{graphicx}
\usepackage{textcomp}
\usepackage{xcolor}
\def\BibTeX{{\rm B\kern-.05em{\sc i\kern-.025em b}\kern-.08em
    T\kern-.1667em\lower.7ex\hbox{E}\kern-.125emX}}

%additional packages
%\usepackage[ngerman]{babel}
\usepackage[utf8]{inputenc} 
\usepackage{hyperref} 
\usepackage{url}   
%%fuer abkuerzungen begin
\usepackage[acronym,hyperfirst = false]{glossaries}
\glsdisablehyper
%\usepackage[acronym,acronymlists={main, abbreviationlist},shortcuts,toc,description,footnote]{glossaries}
\newglossary[clg]{abbreviationlist}{cyi}{cyg}{List of Abbreviations}
\newglossary[slg]{symbolslist}{syi}{syg}{Symbols}
\renewcommand{\firstacronymfont}[1]{\emph{#1}}
\renewcommand*{\glspostdescription}{}	% Punkt am Ende jeder Beschreibung entfernen
%renewcommand*{\acrnameformat}[2]{#2 (\acronymfont{#1})}	% Langform der Akronyme
\makeglossaries
\date{\today}
\newacronym[description={Interface provided by an application or software to be able to implement other programs that uses the software as foundation.}]{API}{API}{Application Programming Interface}

\newacronym[description={In German ``Bundesministerium f\"ur Bildung und Forschung'' is a federal ministry among other things responsible for research projects and the education inside of Germany.}]{BMBF}{BMBF}{Federal Ministry of Education and Research}

\newacronym[description={Hardware component which allows external devices to transfer data directly into or from the memory without using the processor.}]{DMA}{DMA}{Direct Memory Access}

\newacronym[description={Method for storing a number of data items inside of a buffer. Elements which have been stored first will be read first as well.}]{FIFO}{FIFO}{First In, First Out}

\newacronym{FM}{FM}{Frequency Modulation}

\newacronym[description={Stateful network protocol for transfering files from one system to an other system in ``plaintext'' format. FTP requires a reliable, error correcting base protocol like the \gls{TCP}. Besides the ability to read and write files to/from a host, the FTP provides further functionalities like user authorization, directory listing, renaming, or alteration of file access permissions.}]{FTP}{FTP}{File Transfer Protocol}

\newacronym[description={Stateless network protocol built on top of a reliable transport protocol like \gls{TCP} for transfering data from one host to another one. HTTP is the mainly used protocol for web browser interactions with a web server.}]{HTTP}{HTTP}{Hypertext Transfer Protocol}

\newacronym[description={International council responsible for the development and administration of norms in the area of electrical engineering and electronics. Some standards are developed together with the \gls{ISO}.}]{IEC}{IEC}{International Electrotechnical Commission}

\newacronym[description={Primary used communication protocol (in the Internet) for transmitting data packets (or datagrams) from one system to another one. This protocol is located on the Internet layer of the \gls{OSI} and can be split into the older version IPv4 and the newer one IPv6 that allows to address more devices. (cf.\,\cite{Tanenbaum2003})}]{IP}{IP}{Internet Protocol}

\newacronym[description={Set of methods for exchanging data between various threads within a system. In this thesis the term does not include the communication between various physical devices connected via a network to each other.}]{IPC}{IPC}{Inter--Process Communication}

\newacronym[description={International council responsible for the development and administration of norms except in the scope of electrical engineering and electronics. Responsible for these areas is the \gls{IEC}.}]{ISO}{ISO}{International Organization for Standardization}

\newacronym[description={Short function or method that is called as soon as a (specific) interrupt hits the processor core.}]{ISR}{ISR}{Interrupt Service Routine}

\newacronym[description={\ldots}]{MISD}{MISD}{Multi Instruction, Single Data}

\newacronym[description={\ldots}]{MIMD}{MIMD}{Multi Instruction, Multiple Data}

\newacronym[description={The OSI is a design model describing the network stack in seven layers: 1. Physical Layer, 2. Data Link Layer, 3. Network Layer, 4. Transport Layer, 5. Session Layer, 6. Presentation Layer, and 7. Application Layer (sorted from lowest to highest). (cf.\,\cite{Tanenbaum2003}))}]{OSI}{OSI}{Open Systems Interconnection Reference Model}

\newacronym[description={Processor design strategy based on the usage of a little amount of simplified instructions to improve execution performance.}]{RISC}{RISC}{Reduced Instruction Set Computer/Computing}

\newacronym[description={\ldots}]{CISC}{CISC}{Complex Instruction Set Computer/Computing}

\newacronym[description={\ldots}]{SISD}{SISD}{Single Instruction, Single Data}

\newacronym[description={\ldots}]{SIMD}{SIMD}{Single Instruction, Multiple Data}

\newacronym[description={Communication protocol located on the transport layer of the \gls{OSI} that provides a reliable transfer of network packets, i.e. ordered delivery of a byte stream from one program to another one.  The connection between both communication partners is built by a handshaking mechanism. In TCP each received packet is receipted by the recipient. (cf.\,\cite{Tanenbaum2003})}]{TCP}{TCP}{Transmission Control Protocol}

\newacronym[description={Rudimentary version of \gls{FTP} that is built on top of \gls{UDP}. The TFTP implements just the basics of file transfer like read and write. User authorization or directory listing is not provided. Because the protocol is built on top of an unreliable transport protocol a simple packet acknowledgment and checksum verification is used for error correction.}]{TFTP}{TFTP}{Trivial File Transfer Protocol}

\newacronym[description={Communication protocol located on the transport layer of the \gls{OSI} that provides a simple transmission model for packets without the overhead introduced by mechanisms for providing reliability, ordering, or data integrity, like handshaking dialogues. In \gls{UDP} the error detection and error correction is outsourced to higher layers like the application itself. Thus it is assumed that errors either have no influence for successfully processing the use--cases, or the correction is implemented inside of the application's using protocols. (cf.\,\cite{Tanenbaum2003})}]{UDP}{UDP}{User Datagram Protocol}

\newacronym[description={\ldots}]{WCET}{WCET}{Worst Case Execution Time}

\newacronym[description={Text--based data format for application independent and platform independent exchange of information.}]{XML}{XML}{Extensible Markup Language}

\newacronym[plural=ISAs, firstplural=Instruction Set Architectures (ISAs)]{ISA}{ISA}{Instruction Set Architecture}

\newacronym[plural=CPUs, firstplural=Central Processing Units (CPUs)]{CPU}{CPU}{Central Processing Unit}

\newacronym{AES}{AES}{Advanced Encryption Standard}

\newacronym[plural=FPGAs, firstplural=Field-programmable Gate Arrays (FPGAs)]{FPGA}{FPGA}{Field-programmable Gate Array}
%%fuer abkuerzungen end 

        
\begin{document}

\title{Advantages and disadvantages of the RISC-V ISA
(Instruction Set Architecture) in comparison to the
ARMv8 ISA}

\author{\IEEEauthorblockN{1\textsuperscript{st} Michael Schneider}
\IEEEauthorblockA{\textit{Faculity of Computer Science and Mathematics} \\
\textit{OTH Regensburg}\\
Regensburg, Germany \\
michael4.schneider@st.oth-regensburg.de
}
\and
\IEEEauthorblockN{2\textsuperscript{nd} Florian Henneke}
\IEEEauthorblockA{\textit{Faculity of Computer Science and Mathematics} \\
\textit{OTH Regensburg}\\
Regensburg, Germany \\
florian.henneke@st.oth-regensburg.de
}
\and
\IEEEauthorblockN{3\textsuperscript{rd} Alexander Schmid}
\IEEEauthorblockA{\textit{Faculity of Computer Science and Mathematics} \\
\textit{OTH Regensburg} \\
Regensburg, Germany \\
alexander2.schmid@st.oth-regensburg.de}
%\and
%\IEEEauthorblockN{4\textsuperscript{th} Given Name Surname}
%\IEEEauthorblockA{\textit{dept. name of organization (of Aff.)} \\
%\textit{name of organization (of Aff.)}\\
%City, Country \\
%email address}
%\and
%\IEEEauthorblockN{5\textsuperscript{th} Given Name Surname}
%\IEEEauthorblockA{\textit{dept. name of organization (of Aff.)} \\
%\textit{name of organization (of Aff.)}\\
%City, Country \\
%email address}
%\and
%\IEEEauthorblockN{6\textsuperscript{th} Given Name Surname}
%\IEEEauthorblockA{\textit{dept. name of organization (of Aff.)} \\
%\textit{name of organization (of Aff.)}\\
%City, Country \\
%email address}
}

\maketitle

\begin{abstract}
text
\end{abstract}

\begin{IEEEkeywords}
keyword1, keyowrd2
\end{IEEEkeywords}

\section{Introduction}
\label{ref:introduction}
	\subsection{Topic}
	\subsection{Motivation}
	\subsection{Goal}
	\subsection{Overview of paper}

\section{Background}
\label{ref:background}
	\subsection{Instruction Set Architectures}
	\subsection{RISC}
	\subsection{ARM}
	text
	\subsection{RISC-V}
	text

\section{Concept and Methods}
\label{ref:concept}
In order to determine whether RISC-V will gain a significant market share in the following years, it is useful to compare the two \glspl{ISA}
across a set of criteria that are relevant to semiconductor companies when evaluating which \gls{ISA} to use with a new CPU design.

The first of these criteria is the \gls{ISA}'s business model. \glspl{ISA} are often protected by patents that prohibit anyone not licensed
by the patent owner from distributing \glspl{CPU} that implement that \gls{ISA}. \cite{Tang2011} Whether these patents exist and the licensing terms
are an important factor when deciding which \gls{ISA} to use.

The \gls{ISA}'s complexity refers to the amount of effort required to implement the \gls{ISA}. The more complex an \gls{ISA} is, the more developer
time is spent on implementing and verifying a \gls{CPU}'s compatibility to the \gls{ISA}, instead of optimizing the \gls{CPU} for performance and efficiency,
increasing a \gls{CPU}'s development cost.

A \gls{CPU}'s performance can be evaluated under multiple aspects, including the rate of instructions, the rate of floating point operations or
the speed at which the CPU executes a given program. Efficiency considerations, such as the code size of a given program when compiled to the \gls{CPU}'s
instruction set or the amount of power the \gls{CPU} consumes when executing a given program are closely related to performance and shall, for the
purposes of this paper, be grouped under performance.

A program's code size is greatly influenced by the \gls{ISA}, since the instruction set and the compiler used are the only two factors that influence
code size. As such, the code size is an important factor to consider when choosing which \gls{ISA} to implement for a new processor design, especially
for microcontrollers that are usually very constrained in the size of their program memory.
For the other performance aspects, it is debatable to which extent they are influenced by the \gls{CPU}'s instruction set as opposed
to the concrete implementation of the CPU. \cite{Blem2013} \cite{Akram2017}

\glspl{ISA} often allow for a number of instruction set extensions that may or may not be implemented by a given \gls{CPU}. These usually allow
faster and more efficient processing of programs for a given use case, such as \gls{SIMD} extensions that optimize signal processing and media applications,
\gls{AES} extensions that optimize cryptography or \gls{ISA}-extensions with shorter instructions for applications that are constrained in program memory.
Having a small base instruction set with many fine grained extensions improves the flexibility of the \gls{ISA}, allowing \glspl{CPU} to be optimized for specific
use cases, increasing performance and efficiency for those use cases. \gls{ISA} extensions do however pose a disadvantage when distributing precompiled software
to end users, as a piece of software that uses a certain \gls{ISA} extension can't be executed on \glspl{CPU} that don't implement that extension, potentially
increasing the number of different versions of that software that need to be distributed.

An \gls{ISA}'s ecosystem refers to the software that supports that \gls{ISA}, especially compilers that compile to that \gls{ISA}, operating systems and libraries.
When developing a new \gls{CPU} it is preferrable to use an \gls{ISA} with a large ecosystem, in order to maximize the amount of software
that can run of that \gls{CPU}. This is especially important in consumer desktop and mobile devices where a large variety of software is to be executed,
but less important in embedded applications, where the software running on a microcontroller is specifically developed for that application and microcontroller
only.

	\subsection{Business Models}
	When taking a look at the business model of the two rivaling \glspl{ISA} one will detect two substantially different approaches. While ARM takes the traditional path of licensing its intellectual property to semiconductor companies, RISC-V stands out with the completley different way of publishing its \gls{ISA} in an open source manner. This includes giving away their \gls{ISA} definition for free, which raises the standard questions criticizing open source material.

	But let's start at the classic business model: ARM sells its \glspl{ISA} in various licensing models. \cite{ARMLC} These are staggerd in multiple levels of access. The 'Desing Start' level includes free access to the \gls{ISA} Defintion of the simplest ARM Chips Cortex-M0 and Cortex-M3 aswell as a fitting toolchain and processor models. For a fee between \$0 and \$75K you can get the \gls{ISA} of the Cortex-A5 aswell as the permission for 'single use' chip production. This means you are allowed to produce and sell one type of chip for a single purpose e.g. a network controller. For every chip produced a royality must be given to ARM. The 'Design Start' access level also includes a license for accessing a 'artisan physical IP library', a license for universities which includes teaching and prototyping and allows production of own chips without royalities in low margins. At last there is a '\acrshort{FPGA}' license which is free and includes a \gls{FPGA} optimized version of the Cortex-M3 and M1. Production is not allowed in this license.

	The next level of access is called 'Flexible Access' and contains two license models one for \$0 to \$75K which allows one tape-out per year. On top of the entry price one pays per used processor design and a royality per produced chip. The other model starts at \$200K per year and uses the same payment additions as the first one. But it allows unlimited tape-outs and includes employee trainings, design tools and design support.

	Above those access levels, there only officially exits the 'Standard' licensing model. This means one makes a individual contract with ARM.
	Several articles from 2013 \cite{Demerjian2013}\cite{Demerjian2013a} talk about an older licensing model which contains special categories for higher access licenses. The highest of these, often refered as the 'Architectural' license is the only one that allows to edit the \gls{ISA} and develop completely freely. The most prominent companies with such a license are Qualcom which develops and sells mobile phone chips and Apple who just announced a 'Apple Silicon' developed laptop chip on ARM basis. \cite{Apple2020} The article also mentions that preparing a licenese of this form often takes about 6-24 months and states per chip royalities of about 1-2.5\%. It also notices so called 'foundry contracts' where customers can buy silicon ready ARM designs in cooperation with a silicon foundry. Most prominent example here are the Mali GPUs. This offers customers a fast and easy way to expand their chip with for example graphic accelerators.

	Contrary to the ARM license model RISC-V is published using the 'Creative Commons Attribution 4.0' license. \cite{Waterman2017}\cite{Waterman2017a} This license allows to 'share' and 'adapt'. This means you are free to copy and redistribute aswell as modify, change, build upon and sell it commercially. It is not necessary to share changes in an open source way and you are only restricted by giving credit to the original licensor. \cite{CC} An important addition is also, that the license cannot be revoked by the licensor. This means everything about RISC-V that is already published will always be free to use!
	Originally founded by Berkley University the RISC-V \gls{ISA} standard is now managed by the 2015 founded nonprofit organization 'RISC-V International'. \cite{RVIAbout} As the term 'nonprofit' implies, there should be no massive income for the organization. Running expenses and further development of the standard do however require certain liquidity. This is ensured by a membership program surrounding the specification. \cite{RVIMem} Resembling the ARM licensing model, it contains three levels: 'Premier', 'Strategic' and 'Community'. Costing between \$2K and \$250K annually these levels do not restrict access to the \gls{ISA}, but grant several levels of taking influence on the future development of the standard through seats in the 'Technical Steering Commitee', speaker slots on conferences and representation on the official RISC-V International website and blog. There are also three 'Strategic Directors', which are votet out of the 'Premier' and 'Strategic' Members and one Academic aswell as one Communtiy Director, which are votet by the 'Communtiy' level of members. \cite{RVIAss}

	Besides taking incfluence in the development process the membership also includes help in designing \gls{CPU} Cores, teching for employes and more. It also allows the usage of the trademark 'RISC-V'.

	\subsection{Structure and Complexity}
	How many instructions are there? How complex does that make the implementation of a core?
	\subsection{Performance}
	What are the differences in code size? Can we accurately compare the execution speed of both ISAs?
	\subsection{Extensibility}
	What instruction set extensions are there for both ISAs? Who can develop new extensions?
	\subsection{Ecosystem}
	Which compilers support ARM and RISC-V? Which operating systems and libraries?


\section{Discussion}
\label{ref:discussion}
	\subsection{ARM}
	What are the advantages of ARM compared to RISC-V?
	\subsection{RISC-V}
	What are the advantages of RISC-V compared to ARM?
	\subsection{Future directions and challenges}
	How can we more accurately measure performance differences between ARM and RISC-V and how do ISA extensions affect performance?

\section{Conclusion and Outlook}
\label{ref:conclusion}
	\subsection{Summary of results}
	\subsection{Interpretation of results}
	\subsection{Future directions}

\section{Overview of Literature}
Alexander Schmid \cite{Akram2017} \cite{Arm2020} \cite{Asanovic2014} \cite{HeuiLee2001} \cite{Patterson2019} \cite{Perotti2020} \cite{Shore2015} \cite{Waterman2016} \cite{Xu2003}

Florian Henneke \cite{Waterman2016} \cite{Ryzhyk2006} \cite{Asanovic2014} \cite{Furber2000} \cite{Microsoft2020} \cite{Greenwaves2020} \cite{Aws2020} \cite{Microsoft2020}

Michael Schneider \cite{50years} \cite{hennessy2012computer} \cite{drechsler2020enhanced} \cite{WisconsinMadison2016} \cite{IEEE2018} \cite{Dirvin2019} \cite{Bandic2019}

\bibliographystyle{IEEEtran}
\bibliography{bibliography}

\end{document}
