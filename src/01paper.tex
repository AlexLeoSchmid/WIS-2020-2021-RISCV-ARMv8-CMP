\documentclass[conference]{IEEEtran}
\IEEEoverridecommandlockouts
% The preceding line is only needed to identify funding in the first footnote. If that is unneeded, please comment it out.
\usepackage{cite}
\usepackage{amsmath,amssymb,amsfonts}
\usepackage{algorithmic}
\usepackage{graphicx}
\usepackage{textcomp}
\usepackage{xcolor}
\def\BibTeX{{\rm B\kern-.05em{\sc i\kern-.025em b}\kern-.08em
    T\kern-.1667em\lower.7ex\hbox{E}\kern-.125emX}}

%additional packages
%\usepackage[ngerman]{babel}
\usepackage[utf8]{inputenc} 
\usepackage{hyperref} 
\usepackage{url}   
%%fuer abkuerzungen begin
\usepackage[acronym,hyperfirst = false]{glossaries}
\glsdisablehyper
%\usepackage[acronym,acronymlists={main, abbreviationlist},shortcuts,toc,description,footnote]{glossaries}
\newglossary[clg]{abbreviationlist}{cyi}{cyg}{List of Abbreviations}
\newglossary[slg]{symbolslist}{syi}{syg}{Symbols}
\renewcommand{\firstacronymfont}[1]{\emph{#1}}
\renewcommand*{\glspostdescription}{}	% Punkt am Ende jeder Beschreibung entfernen
%renewcommand*{\acrnameformat}[2]{#2 (\acronymfont{#1})}	% Langform der Akronyme
\makeglossaries
\date{\today}

\newacronym[description={Interface between Hard- and Software. Architecture of the Prozessor}]{ISA}{ISA}{Instruction Set Architecture}
\newacronym{RISC}{RISC}{Reduced Instruction Set Computer}
\newacronym{CISC}{CISC}{Complex Instruction Set Computer}

%%fuer abkuerzungen end 

        
\begin{document}

\title{Advantages and disadvantages of the RISC-V ISA
(Instruction Set Architecture) in comparison to the
ARMv8 ISA}

\author{\IEEEauthorblockN{1\textsuperscript{st} Michael Schneider}
\IEEEauthorblockA{\textit{Faculity of Computer Science and Mathematics} \\
\textit{OTH Regensburg}\\
Regensburg, Germany \\
michael4.schneider@st.oth-regensburg.de
}
\and
\IEEEauthorblockN{2\textsuperscript{nd} Florian Henneke}
\IEEEauthorblockA{\textit{Faculity of Computer Science and Mathematics} \\
\textit{OTH Regensburg}\\
Regensburg, Germany \\
florian.henneke@st.oth-regensburg.de
}
\and
\IEEEauthorblockN{3\textsuperscript{rd} Alexander Schmid}
\IEEEauthorblockA{\textit{Faculity of Computer Science and Mathematics} \\
\textit{OTH Regensburg} \\
Regensburg, Germany \\
alexander2.schmid@st.oth-regensburg.de}
%\and
%\IEEEauthorblockN{4\textsuperscript{th} Given Name Surname}
%\IEEEauthorblockA{\textit{dept. name of organization (of Aff.)} \\
%\textit{name of organization (of Aff.)}\\
%City, Country \\
%email address}
%\and
%\IEEEauthorblockN{5\textsuperscript{th} Given Name Surname}
%\IEEEauthorblockA{\textit{dept. name of organization (of Aff.)} \\
%\textit{name of organization (of Aff.)}\\
%City, Country \\
%email address}
%\and
%\IEEEauthorblockN{6\textsuperscript{th} Given Name Surname}
%\IEEEauthorblockA{\textit{dept. name of organization (of Aff.)} \\
%\textit{name of organization (of Aff.)}\\
%City, Country \\
%email address}
}

\maketitle

\begin{abstract}
text
\end{abstract}

\begin{IEEEkeywords}
keyword1, keyowrd2
\end{IEEEkeywords}

\section{Introduction}
\label{ref:introduction}
	\subsection{Topic}
	\subsection{Motivation}
	\subsection{Goal}
	\subsection{Overview of paper}

\section{Background}
\label{ref:background}
	\subsection{Instruction Set Architectures}
	\subsection{RISC}
	\subsection{ARM}
	text
	\subsection{RISC-V}
	text

\section{Concept and Methods (Initial section written by Alexander Schmid)}
\label{ref:concept}
In order to determine whether RISC-V will gain a significant market share in the following years, it is useful to compare the two \glspl{ISA}
across a set of criteria that are relevant to semiconductor companies when evaluating which \gls{ISA} to use with a new CPU design.

The first of these criteria is the \gls{ISA}'s business model. \glspl{ISA} are often protected by patents that prohibit anyone not licensed
by the patent owner from distributing \glspl{CPU} that implement that \gls{ISA}. \cite{Tang2011}

Whether these patents exist and the licensing terms
are an important factor when deciding which \gls{ISA} to use.

The \gls{ISA}'s complexity refers to the amount of effort required to implement the \gls{ISA}. The more complex an \gls{ISA} is, the more developer
time is spent on implementing and verifying a \gls{CPU}'s compatibility to the \gls{ISA}, instead of optimizing the \gls{CPU} for performance and efficiency,
increasing a \gls{CPU}'s development cost. \cite{Patterson1980}

A \gls{CPU}'s performance usually refers to the speed at which the \gls{CPU} executes a given program. Efficiency considerations, such as the code size of a given program
or the amount of power the \gls{CPU} consumes when executing a given program are closely related to performance and shall, for the
purposes of this paper, be grouped under performance.
Code size is largely influenced by the \gls{ISA} and not by the \gls{CPU}'s implementation.
For the other performance aspects, the influence of the \gls{ISA} is debatable. \cite{Blem2013} \cite{Akram2017}

\glspl{ISA} often allow for a number of instruction set extensions that may or may not be implemented by a given \gls{CPU}.
The choice of extensions greatly influence the flexibility of an \gls{ISA} and thus are an important factor to consider which \gls{ISA} to implement for a new \gls{CPU}, because these have a large impact on performance
and development cost of a \gls{CPU}.

An \gls{ISA}'s ecosystem refers to the software that supports that \gls{ISA}, especially compilers that compile to that \gls{ISA}, operating systems and libraries.
When developing a new \gls{CPU} it is preferrable to use an \gls{ISA} with a large ecosystem, in order to maximize the amount of software
that can run of that \gls{CPU}. This is especially important in consumer desktop and mobile devices where a large variety of software is to be executed.

	\subsection{Business Models}
	Who develops the CPU cores, how can you get access to them? Who supports chip manufacturers in designing a chip with that CPU core?
	\subsection{Structure and Complexity}
	How many instructions are there? How complex does that make the implementation of a core?
	\subsection{Performance (written by Alexander Schmid)}
	Since code size is the performance factor most influenced by a \gls{CPU}'s \gls{ISA}, it will be the primary focus of the
	performance comparison.
	Given that code size is most critical in embedded applications, the Embench benchmark suite is a good benchmark
	with which to compare code sizes. It consists of a number of programs frequently used in embedded
	applications, such as CRC, signal filtering, AES and QR code reading. \cite{Patterson2019}
	When compiling the Embench suite for both RV32IMC as well as 32-bit ARM with the Thumb-2 extension using GCC 7,
	the code for RISC-V is approximately 11\% larger than the code for ARM. \cite{Perotti2020}
	Part of this gap in code size can be explained by the relative immaturity of the RISC-V implementation of
	GCC. RISC-V was introduced in 2017 and the code size of the Embench suite
	compiled for RISC-V is lower with each subsequent version of GCC, however still larger than ARM as of 2019. \cite{Patterson2019}

	In \cite{Perotti2020} an extension for RISC-V is introduced, called HCC, that is aimed at reducing the code size of RISC-V.
	This extension brings the code size gap down to 2.2\% for the Embench suite
	and makes the RISC-V code smaller than ARM by 1.75\% in a proprietary IoT benchmark developed by Huawei. \cite{Perotti2020}
	
	TODO: In subsequent submissions, mention RV64 being significantly smaller than AArch64 as described in \cite[page 62]{Waterman2016},
	and possibilities of comparing execution speed and energy efficiency as described in \cite{Blem2013} and \cite{Akram2017}.

	\subsection{Extensibility}
	What instruction set extensions are there for both ISAs? Who can develop new extensions?
	\subsection{Ecosystem}
	Which compilers support ARM and RISC-V? Which operating systems and libraries?


\section{Discussion}
\label{ref:discussion}
	\subsection{Advantages of ARM}
	What are the advantages of ARM compared to RISC-V?
	\subsection{Advantages of RISC-V}
	What are the advantages of RISC-V compared to ARM?
	\subsection{Future directions and challenges}
	How can we more accurately measure performance differences between ARM and RISC-V and how do ISA extensions affect performance?

\section{Conclusion and Outlook}
\label{ref:conclusion}
	\subsection{Summary of results}
	\subsection{Interpretation of results}
	\subsection{Future directions}

\section{Overview of Literature}
Alexander Schmid \cite{Akram2017} \cite{Arm2020} \cite{Asanovic2014} \cite{HeuiLee2001} \cite{Patterson2019} \cite{Perotti2020} \cite{Shore2015} \cite{Waterman2016} \cite{Xu2003}

Florian Henneke \cite{Waterman2016} \cite{Ryzhyk2006} \cite{Asanovic2014} \cite{Furber2000} \cite{Microsoft2020} \cite{Greenwaves2020} \cite{Aws2020} \cite{Microsoft2020}

Michael Schneider \cite{50years} \cite{hennessy2012computer} \cite{drechsler2020enhanced} \cite{WisconsinMadison2016} \cite{IEEE2018} \cite{Dirvin2019} \cite{Bandic2019}

\bibliographystyle{IEEEtran}
\bibliography{bibliography}

\end{document}
