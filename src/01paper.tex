\documentclass[conference]{IEEEtran}
\IEEEoverridecommandlockouts
% The preceding line is only needed to identify funding in the first footnote. If that is unneeded, please comment it out.
\usepackage{cite}
\usepackage{amsmath,amssymb,amsfonts}
\usepackage{algorithmic}
\usepackage{graphicx}
\usepackage{textcomp}
\usepackage{xcolor}
\def\BibTeX{{\rm B\kern-.05em{\sc i\kern-.025em b}\kern-.08em
    T\kern-.1667em\lower.7ex\hbox{E}\kern-.125emX}}

%additional packages
%\usepackage[ngerman]{babel}
\usepackage[utf8]{inputenc} 
\usepackage{hyperref} 
\usepackage{url}   
%%fuer abkuerzungen begin
\usepackage[acronym,hyperfirst = false]{glossaries}
\glsdisablehyper
%\usepackage[acronym,acronymlists={main, abbreviationlist},shortcuts,toc,description,footnote]{glossaries}
\newglossary[clg]{abbreviationlist}{cyi}{cyg}{List of Abbreviations}
\newglossary[slg]{symbolslist}{syi}{syg}{Symbols}
\renewcommand{\firstacronymfont}[1]{\emph{#1}}
\renewcommand*{\glspostdescription}{}	% Punkt am Ende jeder Beschreibung entfernen
\renewcommand*{\acrnameformat}[2]{#2 (\acronymfont{#1})}	% Langform der Akronyme
\makeglossaries
\date{\today}
\newacronym[description={Interface provided by an application or software to be able to implement other programs that uses the software as foundation.}]{API}{API}{Application Programming Interface}

\newacronym[description={In German ``Bundesministerium f\"ur Bildung und Forschung'' is a federal ministry among other things responsible for research projects and the education inside of Germany.}]{BMBF}{BMBF}{Federal Ministry of Education and Research}

\newacronym[description={Hardware component which allows external devices to transfer data directly into or from the memory without using the processor.}]{DMA}{DMA}{Direct Memory Access}

\newacronym[description={Method for storing a number of data items inside of a buffer. Elements which have been stored first will be read first as well.}]{FIFO}{FIFO}{First In, First Out}

\newacronym{FM}{FM}{Frequency Modulation}

\newacronym[description={Stateful network protocol for transfering files from one system to an other system in ``plaintext'' format. FTP requires a reliable, error correcting base protocol like the \gls{TCP}. Besides the ability to read and write files to/from a host, the FTP provides further functionalities like user authorization, directory listing, renaming, or alteration of file access permissions.}]{FTP}{FTP}{File Transfer Protocol}

\newacronym[description={Stateless network protocol built on top of a reliable transport protocol like \gls{TCP} for transfering data from one host to another one. HTTP is the mainly used protocol for web browser interactions with a web server.}]{HTTP}{HTTP}{Hypertext Transfer Protocol}

\newacronym[description={International council responsible for the development and administration of norms in the area of electrical engineering and electronics. Some standards are developed together with the \gls{ISO}.}]{IEC}{IEC}{International Electrotechnical Commission}

\newacronym[description={Primary used communication protocol (in the Internet) for transmitting data packets (or datagrams) from one system to another one. This protocol is located on the Internet layer of the \gls{OSI} and can be split into the older version IPv4 and the newer one IPv6 that allows to address more devices. (cf.\,\cite{Tanenbaum2003})}]{IP}{IP}{Internet Protocol}

\newacronym[description={Set of methods for exchanging data between various threads within a system. In this thesis the term does not include the communication between various physical devices connected via a network to each other.}]{IPC}{IPC}{Inter--Process Communication}

\newacronym[description={International council responsible for the development and administration of norms except in the scope of electrical engineering and electronics. Responsible for these areas is the \gls{IEC}.}]{ISO}{ISO}{International Organization for Standardization}

\newacronym[description={Short function or method that is called as soon as a (specific) interrupt hits the processor core.}]{ISR}{ISR}{Interrupt Service Routine}

\newacronym[description={\ldots}]{MISD}{MISD}{Multi Instruction, Single Data}

\newacronym[description={\ldots}]{MIMD}{MIMD}{Multi Instruction, Multiple Data}

\newacronym[description={The OSI is a design model describing the network stack in seven layers: 1. Physical Layer, 2. Data Link Layer, 3. Network Layer, 4. Transport Layer, 5. Session Layer, 6. Presentation Layer, and 7. Application Layer (sorted from lowest to highest). (cf.\,\cite{Tanenbaum2003}))}]{OSI}{OSI}{Open Systems Interconnection Reference Model}

\newacronym[description={Processor design strategy based on the usage of a little amount of simplified instructions to improve execution performance.}]{RISC}{RISC}{Reduced Instruction Set Computer/Computing}

\newacronym[description={\ldots}]{CISC}{CISC}{Complex Instruction Set Computer/Computing}

\newacronym[description={\ldots}]{SISD}{SISD}{Single Instruction, Single Data}

\newacronym[description={\ldots}]{SIMD}{SIMD}{Single Instruction, Multiple Data}

\newacronym[description={Communication protocol located on the transport layer of the \gls{OSI} that provides a reliable transfer of network packets, i.e. ordered delivery of a byte stream from one program to another one.  The connection between both communication partners is built by a handshaking mechanism. In TCP each received packet is receipted by the recipient. (cf.\,\cite{Tanenbaum2003})}]{TCP}{TCP}{Transmission Control Protocol}

\newacronym[description={Rudimentary version of \gls{FTP} that is built on top of \gls{UDP}. The TFTP implements just the basics of file transfer like read and write. User authorization or directory listing is not provided. Because the protocol is built on top of an unreliable transport protocol a simple packet acknowledgment and checksum verification is used for error correction.}]{TFTP}{TFTP}{Trivial File Transfer Protocol}

\newacronym[description={Communication protocol located on the transport layer of the \gls{OSI} that provides a simple transmission model for packets without the overhead introduced by mechanisms for providing reliability, ordering, or data integrity, like handshaking dialogues. In \gls{UDP} the error detection and error correction is outsourced to higher layers like the application itself. Thus it is assumed that errors either have no influence for successfully processing the use--cases, or the correction is implemented inside of the application's using protocols. (cf.\,\cite{Tanenbaum2003})}]{UDP}{UDP}{User Datagram Protocol}

\newacronym[description={\ldots}]{WCET}{WCET}{Worst Case Execution Time}

\newacronym[description={Text--based data format for application independent and platform independent exchange of information.}]{XML}{XML}{Extensible Markup Language}

\newacronym[plural=ISAs, firstplural=Instruction Set Architectures (ISAs)]{ISA}{ISA}{Instruction Set Architecture}

\newacronym[plural=CPUs, firstplural=Central Processing Units (CPUs)]{CPU}{CPU}{Central Processing Unit}

\newacronym{AES}{AES}{Advanced Encryption Standard}

\newacronym[plural=FPGAs, firstplural=Field-programmable Gate Arrays (FPGAs)]{FPGA}{FPGA}{Field-programmable Gate Array}
%%fuer abkuerzungen end 

        
\begin{document}

\title{A comparison of the ARMv8 and RISC-V Instruction Set Architectures}

\author{\IEEEauthorblockN{1\textsuperscript{st} Michael Schneider}
\IEEEauthorblockA{\textit{Faculity of Computer Science and Mathematics} \\
\textit{OTH Regensburg}\\
Regensburg, Germany \\
michael4.schneider@st.oth-regensburg.de
}
\and
\IEEEauthorblockN{2\textsuperscript{nd} Florian Henneke}
\IEEEauthorblockA{\textit{Faculity of Computer Science and Mathematics} \\
\textit{OTH Regensburg}\\
Regensburg, Germany \\
florian.henneke@st.oth-regensburg.de
}
\and
\IEEEauthorblockN{3\textsuperscript{rd} Alexander Schmid}
\IEEEauthorblockA{\textit{Faculity of Computer Science and Mathematics} \\
\textit{OTH Regensburg} \\
Regensburg, Germany \\
alexander2.schmid@st.oth-regensburg.de}
%\and
%\IEEEauthorblockN{4\textsuperscript{th} Given Name Surname}
%\IEEEauthorblockA{\textit{dept. name of organization (of Aff.)} \\
%\textit{name of organization (of Aff.)}\\
%City, Country \\
%email address}
%\and
%\IEEEauthorblockN{5\textsuperscript{th} Given Name Surname}
%\IEEEauthorblockA{\textit{dept. name of organization (of Aff.)} \\
%\textit{name of organization (of Aff.)}\\
%City, Country \\
%email address}
%\and
%\IEEEauthorblockN{6\textsuperscript{th} Given Name Surname}
%\IEEEauthorblockA{\textit{dept. name of organization (of Aff.)} \\
%\textit{name of organization (of Aff.)}\\
%City, Country \\
%email address}
}

\maketitle

\begin{abstract}
text
\end{abstract}

\begin{IEEEkeywords}
keyword1, keyowrd2
\end{IEEEkeywords}

\section{Introduction}
\label{ref:introduction}
	\subsection{Overview}
	\subsection{Motivation}
	\subsection{Goal}

\section{Background}
\label{ref:background}
	\subsection{Instruction Set Architectures}
	\subsection{RISC}
	\subsection{ARM}
	text
	\subsection{RISC-V}
	text

\section{Concept and Methods}
\label{ref:concept}
	\subsection{Business Models}
	Who develops the CPU cores, how can you get access to them? Who supports chip manufacturers in designing a chip with that CPU core?
	\subsection{Complexity}
	How many instructions are there? How complex does that make the implementation of a core?
	\subsection{Performance}
	What are the differences in code size? Can we accurately compare the execution speed of both ISAs?
	\subsection{Extensibility}
	What instruction set extensions are there for both ISAs? Who can develop new extensions?
	\subsection{Ecosystem}
	Which compilers support ARM and RISC-V? Which operating systems and libraries?


\section{Discussion}
\label{ref:discussion}
	\subsection{ARM}
	What are the advantages of ARM compared to RISC-V?
	\subsection{RISC-V}
	What are the advantages of RISC-V compared to ARM?
	\subsection{Future directions and challenges}
	How can we more accurately measure performance differences between ARM and RISC-V and how do ISA extensions affect performance?

\section{Conclusion and Outlook}
\label{ref:conclusion}
	\subsection{Summary of results}
	\subsection{Accuracy of results}
	\subsection{Future directions}

\section{Overview of Literature}
Alexander Schmid \cite{Akram2017} \cite{Arm2020} \cite{Asanovic2014} \cite{HeuiLee2001} \cite{Patterson2019} \cite{Perotti2020} \cite{Shore2015} \cite{Waterman2016} \cite{Xu2003}

Florian Henneke \cite{Waterman2016} \cite{Ryzhyk2006} \cite{Asanovic2014} \cite{Furber2000} \cite{Microsoft2020} \cite{Greenwaves2020} \cite{Aws2020} \cite{Microsoft2020}

Michael Schneider \cite{50years} \cite{hennessy2012computer} \cite{drechsler2020enhanced} \cite{WisconsinMadison2016} \cite{IEEE2018} \cite{Dirvin2019} \cite{Bandic2019}

\bibliographystyle{IEEEtran}
\bibliography{bibliography}

\end{document}
