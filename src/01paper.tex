\documentclass[conference]{IEEEtran}
\IEEEoverridecommandlockouts
% The preceding line is only needed to identify funding in the first footnote. If that is unneeded, please comment it out.
\usepackage{cite}
\usepackage{amsmath,amssymb,amsfonts}
\usepackage{algorithmic}
\usepackage{graphicx}
\usepackage{textcomp}
\usepackage{xcolor}
\def\BibTeX{{\rm B\kern-.05em{\sc i\kern-.025em b}\kern-.08em
    T\kern-.1667em\lower.7ex\hbox{E}\kern-.125emX}}

%additional packages
%\usepackage[ngerman]{babel}
\usepackage[utf8]{inputenc} 
\usepackage{hyperref} 
\usepackage{url}   
%%fuer abkuerzungen begin
\usepackage[acronym,hyperfirst = false]{glossaries}
\glsdisablehyper
%\usepackage[acronym,acronymlists={main, abbreviationlist},shortcuts,toc,description,footnote]{glossaries}
\newglossary[clg]{abbreviationlist}{cyi}{cyg}{List of Abbreviations}
\newglossary[slg]{symbolslist}{syi}{syg}{Symbols}
\renewcommand{\firstacronymfont}[1]{\emph{#1}}
\renewcommand*{\glspostdescription}{}	% Punkt am Ende jeder Beschreibung entfernen
\renewcommand*{\acrnameformat}[2]{#2 (\acronymfont{#1})}	% Langform der Akronyme
\makeglossaries
\date{\today}
\newacronym[description={Interface provided by an application or software to be able to implement other programs that uses the software as foundation.}]{API}{API}{Application Programming Interface}

\newacronym[description={In German ``Bundesministerium f\"ur Bildung und Forschung'' is a federal ministry among other things responsible for research projects and the education inside of Germany.}]{BMBF}{BMBF}{Federal Ministry of Education and Research}

\newacronym[description={Hardware component which allows external devices to transfer data directly into or from the memory without using the processor.}]{DMA}{DMA}{Direct Memory Access}

\newacronym[description={Method for storing a number of data items inside of a buffer. Elements which have been stored first will be read first as well.}]{FIFO}{FIFO}{First In, First Out}

\newacronym{FM}{FM}{Frequency Modulation}

\newacronym[description={Stateful network protocol for transfering files from one system to an other system in ``plaintext'' format. FTP requires a reliable, error correcting base protocol like the \gls{TCP}. Besides the ability to read and write files to/from a host, the FTP provides further functionalities like user authorization, directory listing, renaming, or alteration of file access permissions.}]{FTP}{FTP}{File Transfer Protocol}

\newacronym[description={Stateless network protocol built on top of a reliable transport protocol like \gls{TCP} for transfering data from one host to another one. HTTP is the mainly used protocol for web browser interactions with a web server.}]{HTTP}{HTTP}{Hypertext Transfer Protocol}

\newacronym[description={International council responsible for the development and administration of norms in the area of electrical engineering and electronics. Some standards are developed together with the \gls{ISO}.}]{IEC}{IEC}{International Electrotechnical Commission}

\newacronym[description={Primary used communication protocol (in the Internet) for transmitting data packets (or datagrams) from one system to another one. This protocol is located on the Internet layer of the \gls{OSI} and can be split into the older version IPv4 and the newer one IPv6 that allows to address more devices. (cf.\,\cite{Tanenbaum2003})}]{IP}{IP}{Internet Protocol}

\newacronym[description={Set of methods for exchanging data between various threads within a system. In this thesis the term does not include the communication between various physical devices connected via a network to each other.}]{IPC}{IPC}{Inter--Process Communication}

\newacronym[description={International council responsible for the development and administration of norms except in the scope of electrical engineering and electronics. Responsible for these areas is the \gls{IEC}.}]{ISO}{ISO}{International Organization for Standardization}

\newacronym[description={Short function or method that is called as soon as a (specific) interrupt hits the processor core.}]{ISR}{ISR}{Interrupt Service Routine}

\newacronym[description={\ldots}]{MISD}{MISD}{Multi Instruction, Single Data}

\newacronym[description={\ldots}]{MIMD}{MIMD}{Multi Instruction, Multiple Data}

\newacronym[description={The OSI is a design model describing the network stack in seven layers: 1. Physical Layer, 2. Data Link Layer, 3. Network Layer, 4. Transport Layer, 5. Session Layer, 6. Presentation Layer, and 7. Application Layer (sorted from lowest to highest). (cf.\,\cite{Tanenbaum2003}))}]{OSI}{OSI}{Open Systems Interconnection Reference Model}

\newacronym[description={Processor design strategy based on the usage of a little amount of simplified instructions to improve execution performance.}]{RISC}{RISC}{Reduced Instruction Set Computer/Computing}

\newacronym[description={\ldots}]{CISC}{CISC}{Complex Instruction Set Computer/Computing}

\newacronym[description={\ldots}]{SISD}{SISD}{Single Instruction, Single Data}

\newacronym[description={\ldots}]{SIMD}{SIMD}{Single Instruction, Multiple Data}

\newacronym[description={Communication protocol located on the transport layer of the \gls{OSI} that provides a reliable transfer of network packets, i.e. ordered delivery of a byte stream from one program to another one.  The connection between both communication partners is built by a handshaking mechanism. In TCP each received packet is receipted by the recipient. (cf.\,\cite{Tanenbaum2003})}]{TCP}{TCP}{Transmission Control Protocol}

\newacronym[description={Rudimentary version of \gls{FTP} that is built on top of \gls{UDP}. The TFTP implements just the basics of file transfer like read and write. User authorization or directory listing is not provided. Because the protocol is built on top of an unreliable transport protocol a simple packet acknowledgment and checksum verification is used for error correction.}]{TFTP}{TFTP}{Trivial File Transfer Protocol}

\newacronym[description={Communication protocol located on the transport layer of the \gls{OSI} that provides a simple transmission model for packets without the overhead introduced by mechanisms for providing reliability, ordering, or data integrity, like handshaking dialogues. In \gls{UDP} the error detection and error correction is outsourced to higher layers like the application itself. Thus it is assumed that errors either have no influence for successfully processing the use--cases, or the correction is implemented inside of the application's using protocols. (cf.\,\cite{Tanenbaum2003})}]{UDP}{UDP}{User Datagram Protocol}

\newacronym[description={\ldots}]{WCET}{WCET}{Worst Case Execution Time}

\newacronym[description={Text--based data format for application independent and platform independent exchange of information.}]{XML}{XML}{Extensible Markup Language}

\newacronym[plural=ISAs, firstplural=Instruction Set Architectures (ISAs)]{ISA}{ISA}{Instruction Set Architecture}

\newacronym[plural=CPUs, firstplural=Central Processing Units (CPUs)]{CPU}{CPU}{Central Processing Unit}

\newacronym{AES}{AES}{Advanced Encryption Standard}

\newacronym[plural=FPGAs, firstplural=Field-programmable Gate Arrays (FPGAs)]{FPGA}{FPGA}{Field-programmable Gate Array}
%%fuer abkuerzungen end 

        
\begin{document}

\title{ARMv8 advantages and disadvantages to RISC-V}

\maketitle

\begin{abstract}
text
\end{abstract}

\begin{IEEEkeywords}
keyword1, keyowrd2
\end{IEEEkeywords}

\section{Introduction}
\label{ref:introduction}
	\subsection{Overview}
	\subsection{Motivation}
	\subsection{Goal}

\section{Background}
\label{ref:background}
	\subsection{Instruction Set Architectures}
	\subsection{RISC}
	\subsection{ARM}
	text
	\subsection{RISC-V}
	text

\section{Concept and Methods}
\label{ref:concept}
	\subsection{Business Models}
	Who develops the CPU cores, how can you get access to them? Who supports chip manufacturers in designing a chip with that CPU core?
	\subsection{Complexity}
	Risc-V and ARMv8 are both \gls{RISC} based architectures and copmared to \gls{CISC} they are much less complex. \gls{RISC} machines are having a lot of charactaristics, which are not necessary to be completely implemented, but they can leed to a few advantages of \gls{RISC} compared to \gls{CISC},  like an simpler control unit and faster decode, both because of the less instructions and addressing modes. \cite{George1990}\\
	Even various \gls{RISC} \gls{ISA} are different in complexity. To compare those differents in complexity, it is important to take a look into the RISC specific features and compare them in flexibility, amount and implementation.
	\subsubsection{Instruction sets}
	While the ARMv8 is clearly defined in relation to the instruction set, RISC-V is there much more variable. In RISC-V the only necessary Instruction set is the Integer instruction set. Those base integer instructions cannot be redened, only extended by more optional instruction-sets. Additional to operations for the base Integer Instruction Set, completely different extensions are available.
	\begin{description}
	\item[M]	Extension for Integer Multiplication and Division
	\item[A]	Extension for Atomic Instructions
	\item[F]	Extension for Single-Precision Floating-Point
	\item[D]	Extension for Double-Precision Floating-Point
	\item[Q]	Extension for Quad-Precision Floating-Point
	\item[L]	Extension for Decimal Floating-Point
	\item[C] 	Extension for Compressed Instructions
	\item[V]	Extension for Vector Operations
	\item[B]	Extension for Bit Manipulation
	\item[T]	Extension for Transactional Memory
	\item[P]	Extension for Packet-SIMD Instructions
	\end{description}
The added alphabetical characters are defining the used instruction set with all his implemented extension. While an ''M'' marks the architecture to be able to multiply the basic representation of on integer an extension like ''F'' or ''V'' makes the architecture undestanding a completely new number representation. This makes the RISC-V architecture only as complex as necessary because all the gratuitous instructions are not implemented. \cite{Asanovic2016} \\
ARM insteed is defining the ARMv8 architecture in a completely different way. The ARMv8 supports also those extensions which are mentioned above but already in the basic version, which is also called v8.0. To build a more complex CPU the ARMv8 also provides extensions like ARMV8.1, ... , ARMv8.6 plus a few optional extensions.  Almost all the optional extensions of RISC-V are covered by the basic v8.0, which makes the ARMv8 only from the instructions set of view as complex as a fully extended RISC-V architecture. All those additional extensions like v8.1 etc makes the architecture much more complex than the RISC-V full extended one. \cite{ArmManual} \\

\subsubsection{Instruction set implemtations}
Additonal to the basic instruction sets in the chapter above the different \gls{ISA} are able to implement those basic or extension instructions in different ways. Again the start will be done by the RISC-V architecture. This architecture is able to implement the instruction sets in 3 different versions a 32-bit (RV32I), a 64-bit (RV64I) and a 128-bit version (RV128I).
While for the 32-bit and the 64-bit implementations are again multiple subversions available, RV32E, RV32G/RV64G, the RV128I is the only 128-bit implementation so far. The, lets say default version, RV32I implements 31 integer registers, the RV32E instead implements only 15 integer register. The subversion RV32G/RV64G is less a own version of implementation than a stable release. The RV32G/RV64G is not defining registers or instruction sets in its one way, it is combining a basic \gls{ISA} (RV32I or RV64I) plus different selected standard extensions (IMAFD). \cite{Asanovic2016} \\
Also ARMv8 has, as aspected, different implementations, as well. While RISC-V is defining the implementations based on a number with the extensions, which are implemented, ARMv8 does it in a different way, because there are all the standard extensions already in the basic version included.
ARMv8 provides 3 different implementations A64, A32 and T32. As mentioned A64 is the 64-bit version and A32, T32  are both 32-bit versions.
AARCH64 and AARCH32 are two different execution states in ARM (AARCH64 for 64-bit and AARCH32 for 32-bit), which are supporting the A64 in AARCH64 or A32/T32 in AARCH32.\cite{ArmManual}\\

Write some stuff comparing machine-level and supervisor level \cite{AndrewWaterman} with 3 basic architecture profiles (Application profil, real-time profil, microcontroller profil\cite{ArmManual} S. 36

\subsubsection{registers and access}
To compare the overview in an maybe for users more abstract point of few, the basic assembler commands (load and read), which are always necessary to read and write the registers of an \gls{RISC} architecture. Taking an basic RISC-V architecture RV32I for example, which has 32 Registers from x0 to x31. Each of those registers is 32 bits (1 word) wide. 
To load an value from a register lw or lb is called. lw means load word, which loads the complete register and lb means load byte which is an alternativ way in RISC-V to load only more specific parameters. Both are built-on the same way. Both are taking 2 parameters, a temp register to store the value in and an adress to load the value from. the address is a pointer to the register together with an offset.
In the same way the store instructions sw or sb are working. sw means store word and sb means store byte. The built-on is the same way as it is for lw and lb. But the left side of the command is the temp register to take the value from and the second parameter is the register to store the value in (again with an offset) \cite{RISC-Assembly}
ARM instead has only two instructions one to load (LDR) and another one to store (STR) values. But the instructions have subcalls (STRB, STRH, STRW) as well as the (LDR x0, LDR w0,) 
%% godbolt.org
%% basic comparing between basic instructions like load, ...

	To compare the both \glspl{ISA} (RISC-V and ARMv8), which are the main components in this paper. I will base the argumentation on the amount of Instructions, the different modes 		of addressing and the complexitiy (how many different things are claryfied) in the manuel of each \gls{ISA}. 
	Only by counting the pages of the ARM manule (xxx) compared with the pages of the RISC-V manuel (xxx) we can assume that ARM is much more complex than RISC-V. By comparing 
	- Instructions starts on \cite{ArmManual} S. 222


	How many instructions are there? How complex does that make the implementation of a core?
	\subsection{Performance}
	What are the differences in code size? Can we accurately compare the execution speed of both ISAs?
	\subsection{Extensibility}
	What instruction set extensions are there for both ISAs? Who can develop new extensions?
	\subsection{Ecosystem}
	Which compilers support ARM and RISC-V? Which operating systems and libraries?


\section{Discussion}
\label{ref:discussion}
	\subsection{ARM}
	What are the advantages of ARM compared to RISC-V?
	\subsection{RISC-V}
	What are the advantages of RISC-V compared to ARM?
	\subsection{Future directions and challenges}
	How can we more accurately measure performance differences between ARM and RISC-V and how do ISA extensions affect performance?

\section{Conclusion and Outlook}
\label{ref:conclusion}
	\subsection{Summary of results}
	\subsection{Accuracy of results}
	\subsection{Future directions}

\subsection{Literaturverzeichnis}
Die Quellen befinden sich in der Datei \textit{biblography.bib}. 
Meine Quellen sind:l \cite{50years}, \cite{ArmManual}, \cite{hennessy2012computer}, \cite{WisconsinMadison2016} \cite{drechsler2020enhanced}, \cite{Asanovic2016}, \cite{IEEE2018} \cite{Dirvin2019}, \cite{Bandic2019}.

\bibliographystyle{IEEEtran}
\bibliography{bibliography}

\end{document}
