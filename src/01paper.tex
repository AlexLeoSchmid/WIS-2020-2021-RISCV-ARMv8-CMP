\documentclass[conference]{IEEEtran}
\IEEEoverridecommandlockouts
% The preceding line is only needed to identify funding in the first footnote. If that is unneeded, please comment it out.
\usepackage{cite}
\usepackage{amsmath,amssymb,amsfonts}
\usepackage{algorithmic}
\usepackage{graphicx}
\usepackage{textcomp}
\usepackage{xcolor}
\def\BibTeX{{\rm B\kern-.05em{\sc i\kern-.025em b}\kern-.08em
    T\kern-.1667em\lower.7ex\hbox{E}\kern-.125emX}}

%additional packages
%\usepackage[ngerman]{babel}
\usepackage[utf8]{inputenc} 
\usepackage{hyperref} 
\usepackage{url}   
%%fuer abkuerzungen begin
\usepackage[acronym,hyperfirst = false]{glossaries}
\glsdisablehyper
%\usepackage[acronym,acronymlists={main, abbreviationlist},shortcuts,toc,description,footnote]{glossaries}
\newglossary[clg]{abbreviationlist}{cyi}{cyg}{List of Abbreviations}
\newglossary[slg]{symbolslist}{syi}{syg}{Symbols}
\renewcommand{\firstacronymfont}[1]{\emph{#1}}
\renewcommand*{\glspostdescription}{}	% Punkt am Ende jeder Beschreibung entfernen
%renewcommand*{\acrnameformat}[2]{#2 (\acronymfont{#1})}	% Langform der Akronyme
\makeglossaries
\date{\today}
\newacronym[description={Interface provided by an application or software to be able to implement other programs that uses the software as foundation.}]{API}{API}{Application Programming Interface}

\newacronym[description={In German ``Bundesministerium f\"ur Bildung und Forschung'' is a federal ministry among other things responsible for research projects and the education inside of Germany.}]{BMBF}{BMBF}{Federal Ministry of Education and Research}

\newacronym[description={Hardware component which allows external devices to transfer data directly into or from the memory without using the processor.}]{DMA}{DMA}{Direct Memory Access}

\newacronym[description={Method for storing a number of data items inside of a buffer. Elements which have been stored first will be read first as well.}]{FIFO}{FIFO}{First In, First Out}

\newacronym{FM}{FM}{Frequency Modulation}

\newacronym[description={Stateful network protocol for transfering files from one system to an other system in ``plaintext'' format. FTP requires a reliable, error correcting base protocol like the \gls{TCP}. Besides the ability to read and write files to/from a host, the FTP provides further functionalities like user authorization, directory listing, renaming, or alteration of file access permissions.}]{FTP}{FTP}{File Transfer Protocol}

\newacronym[description={Stateless network protocol built on top of a reliable transport protocol like \gls{TCP} for transfering data from one host to another one. HTTP is the mainly used protocol for web browser interactions with a web server.}]{HTTP}{HTTP}{Hypertext Transfer Protocol}

\newacronym[description={International council responsible for the development and administration of norms in the area of electrical engineering and electronics. Some standards are developed together with the \gls{ISO}.}]{IEC}{IEC}{International Electrotechnical Commission}

\newacronym[description={Primary used communication protocol (in the Internet) for transmitting data packets (or datagrams) from one system to another one. This protocol is located on the Internet layer of the \gls{OSI} and can be split into the older version IPv4 and the newer one IPv6 that allows to address more devices. (cf.\,\cite{Tanenbaum2003})}]{IP}{IP}{Internet Protocol}

\newacronym[description={Set of methods for exchanging data between various threads within a system. In this thesis the term does not include the communication between various physical devices connected via a network to each other.}]{IPC}{IPC}{Inter--Process Communication}

\newacronym[description={International council responsible for the development and administration of norms except in the scope of electrical engineering and electronics. Responsible for these areas is the \gls{IEC}.}]{ISO}{ISO}{International Organization for Standardization}

\newacronym[description={Short function or method that is called as soon as a (specific) interrupt hits the processor core.}]{ISR}{ISR}{Interrupt Service Routine}

\newacronym[description={\ldots}]{MISD}{MISD}{Multi Instruction, Single Data}

\newacronym[description={\ldots}]{MIMD}{MIMD}{Multi Instruction, Multiple Data}

\newacronym[description={The OSI is a design model describing the network stack in seven layers: 1. Physical Layer, 2. Data Link Layer, 3. Network Layer, 4. Transport Layer, 5. Session Layer, 6. Presentation Layer, and 7. Application Layer (sorted from lowest to highest). (cf.\,\cite{Tanenbaum2003}))}]{OSI}{OSI}{Open Systems Interconnection Reference Model}

\newacronym[description={Processor design strategy based on the usage of a little amount of simplified instructions to improve execution performance.}]{RISC}{RISC}{Reduced Instruction Set Computer/Computing}

\newacronym[description={\ldots}]{CISC}{CISC}{Complex Instruction Set Computer/Computing}

\newacronym[description={\ldots}]{SISD}{SISD}{Single Instruction, Single Data}

\newacronym[description={\ldots}]{SIMD}{SIMD}{Single Instruction, Multiple Data}

\newacronym[description={Communication protocol located on the transport layer of the \gls{OSI} that provides a reliable transfer of network packets, i.e. ordered delivery of a byte stream from one program to another one.  The connection between both communication partners is built by a handshaking mechanism. In TCP each received packet is receipted by the recipient. (cf.\,\cite{Tanenbaum2003})}]{TCP}{TCP}{Transmission Control Protocol}

\newacronym[description={Rudimentary version of \gls{FTP} that is built on top of \gls{UDP}. The TFTP implements just the basics of file transfer like read and write. User authorization or directory listing is not provided. Because the protocol is built on top of an unreliable transport protocol a simple packet acknowledgment and checksum verification is used for error correction.}]{TFTP}{TFTP}{Trivial File Transfer Protocol}

\newacronym[description={Communication protocol located on the transport layer of the \gls{OSI} that provides a simple transmission model for packets without the overhead introduced by mechanisms for providing reliability, ordering, or data integrity, like handshaking dialogues. In \gls{UDP} the error detection and error correction is outsourced to higher layers like the application itself. Thus it is assumed that errors either have no influence for successfully processing the use--cases, or the correction is implemented inside of the application's using protocols. (cf.\,\cite{Tanenbaum2003})}]{UDP}{UDP}{User Datagram Protocol}

\newacronym[description={\ldots}]{WCET}{WCET}{Worst Case Execution Time}

\newacronym[description={Text--based data format for application independent and platform independent exchange of information.}]{XML}{XML}{Extensible Markup Language}

\newacronym[plural=ISAs, firstplural=Instruction Set Architectures (ISAs)]{ISA}{ISA}{Instruction Set Architecture}

\newacronym[plural=CPUs, firstplural=Central Processing Units (CPUs)]{CPU}{CPU}{Central Processing Unit}

\newacronym{AES}{AES}{Advanced Encryption Standard}

\newacronym[plural=FPGAs, firstplural=Field-programmable Gate Arrays (FPGAs)]{FPGA}{FPGA}{Field-programmable Gate Array}
%%fuer abkuerzungen end 

        
\begin{document}

\title{Advantages and disadvantages of the RISC-V ISA
(Instruction Set Architecture) in comparison to the
ARMv8 ISA}

\author{\IEEEauthorblockN{1\textsuperscript{st} Michael Schneider}
\IEEEauthorblockA{\textit{Faculity of Computer Science and Mathematics} \\
\textit{OTH Regensburg}\\
Regensburg, Germany \\
michael4.schneider@st.oth-regensburg.de
}
\and
\IEEEauthorblockN{2\textsuperscript{nd} Florian Henneke}
\IEEEauthorblockA{\textit{Faculity of Computer Science and Mathematics} \\
\textit{OTH Regensburg}\\
Regensburg, Germany \\
florian.henneke@st.oth-regensburg.de
}
\and
\IEEEauthorblockN{3\textsuperscript{rd} Alexander Schmid}
\IEEEauthorblockA{\textit{Faculity of Computer Science and Mathematics} \\
\textit{OTH Regensburg} \\
Regensburg, Germany \\
alexander2.schmid@st.oth-regensburg.de}
%\and
%\IEEEauthorblockN{4\textsuperscript{th} Given Name Surname}
%\IEEEauthorblockA{\textit{dept. name of organization (of Aff.)} \\
%\textit{name of organization (of Aff.)}\\
%City, Country \\
%email address}
%\and
%\IEEEauthorblockN{5\textsuperscript{th} Given Name Surname}
%\IEEEauthorblockA{\textit{dept. name of organization (of Aff.)} \\
%\textit{name of organization (of Aff.)}\\
%City, Country \\
%email address}
%\and
%\IEEEauthorblockN{6\textsuperscript{th} Given Name Surname}
%\IEEEauthorblockA{\textit{dept. name of organization (of Aff.)} \\
%\textit{name of organization (of Aff.)}\\
%City, Country \\
%email address}
}

\maketitle

\begin{abstract}
text
\end{abstract}

\begin{IEEEkeywords}
keyword1, keyowrd2
\end{IEEEkeywords}

\section{Introduction}
\label{ref:introduction}
	\subsection{Topic}
	\subsection{Motivation}
	\subsection{Goal}
	\subsection{Overview of paper}

\section{Background}
\label{ref:background}
	\subsection{Instruction Set Architectures}
	\subsection{RISC}
	\subsection{ARM}
	text
	\subsection{RISC-V}
	text

\section{Concept and Methods (Initial section written by Alexander Schmid)}
\label{ref:concept}

Given that the goal of this paper is to compare the two \glspl{ISA}
across a set of criteria that are relevant to semiconductor companies when evaluating which \gls{ISA} to use with a new CPU design,
these criteria shall be defined in the following section.

The first of these criteria is the \gls{ISA}'s business model. \glspl{ISA} are often protected by patents that prohibit anyone not licensed
by the patent owner from distributing \glspl{CPU} that implement that \gls{ISA}. \cite{Tang2011}

Whether these patents exist and the licensing terms
are an important factor when deciding which \gls{ISA} to use.

The \gls{ISA}'s complexity refers to the amount of effort required to implement the \gls{ISA}. The more complex an \gls{ISA} is, the more developer
time is spent on implementing and verifying the implementation of the \gls{ISA}, instead of optimizing the \gls{CPU} for performance and efficiency,
increasing a \gls{CPU}'s development cost. \cite{Patterson1980}

A \gls{CPU}'s performance usually refers to the speed at which the \gls{CPU} executes a given program. Efficiency considerations, such as the code size of a given program
or the amount of power the \gls{CPU} consumes when executing a given program are closely related to performance and shall, for the
purposes of this paper, be grouped under performance.


\glspl{ISA} often allow for a number of instruction set extensions that may or may not be implemented by a given \gls{CPU}.
The choice of extensions greatly influence the flexibility of an \gls{ISA}. Thus they are an important factor to consider which \gls{ISA} to implement for a new \gls{CPU}, because these have a large impact on performance
and development cost of a \gls{CPU}.

An \gls{ISA}'s ecosystem refers to the software that supports that \gls{ISA}, especially compilers that compile to that \gls{ISA}, operating systems and libraries.
When developing a new \gls{CPU} it is preferrable to use an \gls{ISA} with a large ecosystem, in order to maximize the amount of software
that can run of that \gls{CPU}. This is especially important in consumer desktop and mobile devices where a large variety of software is to be executed.

	\subsection{Business Models (written by Florian Henneke)}
	%When taking a look at the business model of the two rivaling \glspl{ISA} one will detect two substantially different approaches. While ARM takes the traditional path of licensing its intellectual property to semiconductor companies, RISC-V stands out with the completely different way of publishing its \gls{ISA} in an open source manner. This includes giving away their \gls{ISA} definition for free, which raises the standard questions criticizing open source material.
	[Introduction shortend because of word limit]

	Beginning with the classic model of ARM, the following sections will cover the two license models: ARM sells its \glspl{ISA} in various licensing models. \cite{ARMLC} These are staggered in multiple levels of access. The 'Design Start' level includes free access to the IP-Core of the simplest ARM Chips Cortex-M0 and Cortex-M3 aswell as the corresponding toolchain and processor models. For a fee between \$0 and \$75K the IP-Core of the Cortex-A5 aswell as the permission for 'single use' chip production can be aquired. This allows the customer to produce and sell one type of chip for a single purpose e.g. a network controller. For every chip produced, a royality must be given to ARM. The 'Design Start' access level also includes a license for accessing a 'artisan physical IP library', a license for universities which includes teaching and prototyping and allows non commercial production of own chips without royalities in small volumes. At last there is an '\acrshort{FPGA}' license which is free and includes a \gls{FPGA} optimized version of the Cortex-M3 and M1. Production is not allowed with this license.

	%The next level of access is called 'Flexible Access' and contains two license models, one for \$0 to \$75K which allows one tape-out per year. On top of the entry price one pays per used processor design and a royality per produced chip. The other model starts at \$200K per year and uses the same payment additions as the first one. But it allows unlimited tape-outs and includes employee trainings, design tools and design support.
	[Removed short section due to word limit]

	Above those access levels, there only officially exists the 'Standard' licensing model. This means one makes a individual contract with ARM.
	Several articles from 2013 \cite{Demerjian2013}\cite{Demerjian2013a} talk about an older licensing model which contains special categories for higher access licenses. The highest of these, often referred as the 'Architectural' license is the only one that allows editing of the \gls{ISA} and developing completely freely. The most prominent companies with such a license are Qualcomm which develops and sells mobile phone chips and Apple who just announced a 'Apple Silicon' developed laptop chip based on ARM. \cite{Apple2020} The article also mentions that preparing a license of this form often takes about 6-24 months and states per-chip royalities of about 1-2.5\%. It also notices so called 'foundry contracts' where customers can buy silicon ready ARM designs in cooperation with a silicon foundry. The most prominent example here are the Mali GPUs. This offers customers a fast and easy way to expand their chip with, for example, graphic accelerators.

	In contrast to the ARM license model, RISC-V is published using the 'Creative Commons Attribution 4.0' license. \cite{Waterman2017}\cite{Waterman2017a} This license allows the user to 'share' and 'adapt'. This means you are free to copy and redistribute as well as modify, change, build upon and sell it commercially. It is not necessary to share changes in an open source manner and you are only restricted by the obligation to give credit to the original licensor. \cite{CC} An important addition is also, that the license cannot be revoked by the licensor. This means everything about RISC-V that is already published will always be free to use.
	Originally founded by Berkley University, the RISC-V \gls{ISA} standard is now managed by the nonprofit organization 'RISC-V International', founded in 2015. \cite{RVIAbout} As the statutes of the organization include, the association has 'no pecunuary, self-help or commercial purpose' \cite{RVIArt}. Running expenses and further development of the standard do however require a certain liquidity. This is ensured by a membership program surrounding the specification. \cite{RVIMem} Resembling the ARM licensing model, it contains three levels: 'Premier', 'Strategic' and 'Community'. Costing between \$2K and \$250K annually, these levels do not restrict access to the \gls{ISA}, but grant several levels of taking influence on the future development of the standard through seats in the 'Technical Steering Commitee', speaker slots on conferences and representation on the official RISC-V International website and blog. There are also three 'Strategic Directors', which are elected out of the 'Premier' and 'Strategic' Members and one Academic as well as one Communtiy Director, which is elected by the 'Communtiy' level of members. \cite{RVIAss}

	Besides taking influence in the development process, the membership also includes help in designing \gls{CPU} Cores, teaching for employees and more. It also allows the usage of the trademark 'RISC-V'.

	\subsection{Complexity (Michael Schneider)}
	RISC-V and ARMv8 are both \gls{RISC} based architectures. %Compared to \gls{CISC} based architectures they are less powerful in their individual instructions and they are operating on register. \gls{RISC} machines have a lot of characteristics, which are not necessary to be completely implemented, but they can lead to different advantages of \gls{RISC}, if they are, like a simpler control unit and faster decoding. \cite{George1990} \\
	Various \gls{RISC} \glspl{ISA} are different in complexity. To compare those differences, the basic instruction sets with corresponding extensions, the different realisations and two basic assembly instructions will be covered in the next chapters.
	
	\subsubsection{Instruction sets}
	The extension concept of the RISC-V architecture makes it only as complex as necessary, because the \gls{ISA} can be tailored to the specific needs. In RISC-V the only mandatory instruction set is the Integer instruction set. Those base integer instructions cannot be redefined, only extended by optional instruction sets. \cite{Waterman2017} Further details about the extensions are in chapter \ref{sec:extensibility}. \\
	ARM instead defines the ARMv8 architecture in a completely different way. The ARMv8 already supports many more extensions in the basic version, also called v8.0. Further extensions are available in later versions, as explained in the chapter \ref{sec:extensibility}. \cite{Arm2020} \\
	Because almost all the optional extensions of RISC-V are covered by the basic v8.0, the ARMv8, regarding only the instructions, is at least as complex as a fully extended RISC-V architecture. 
	%All those additional extensions like v8.1 etc. makes the architecture much more complex than the full extended RISC-V.
	
	\subsubsection{Instruction set implementations}
	The different \glspl{ISA} are able to implement the explained instruction sets in various ways.
	The RISC-V architecture is able to implement the instruction sets in 3 different word length, a 32-bit (RV32I), a 64-bit (RV64I) and a 128-bit version (RV128I).
	For the 32-bit and the 64-bit implementations are also multiple subversions available, RV32E and RV32G/RV64G. For 128-bit, RV128I is the only 128-bit implementation so far. RV32E is a version with only 15 instead of 31 registers. The subversion RV32G/RV64G is less a own version than a stable release. The RV32G/RV64G is combining a basic \gls{ISA} (RV32I or RV64I) plus different selected standard extensions (IMAFD). \cite{Waterman2016} \\
	Also ARMv8 has, different implementations,
	A64, A32 and T32. A64 is the 64-bit version and A32, T32  are both 32-bit versions.
	AArch64 and AArch32 are two different execution states in ARM (AArch64 for 64-bit and AArch32 for 32-bit). These execution states support the A64 instruction set in AArch64 and A32 and T32 in AArch32. While A32 uses fixed 32-bit instruction encodings, T32 is variable, which means it uses both 16-bit and 32-bit instructions. This is good to optimize your code, because a common instruction selection rule says, if a 16-bit encoding and a 32-bit encoding are available, the 16-bit is the one you should use, to optimize the code density. \cite{Arm2020}\\
	
	\subsubsection{registers and access}
	To complete the overview in a, for users more abstract point of view, two basic assembler commands (load and store) and the different instruction formats are compared.
	To load a value from a RISC-V register, LW, LH or LB is used. The "L" means load and the following characters stand for word (32 bit), halfword (16 bit) and byte. LH and LB are signed and can be extended by an "U" (LHU, LBU) to load unsigned values. \cite{Waterman2017}
	All instructions take 2 parameters, a register to store the value in and an address to load the value from. The address consists of the value stored in a register with an immediate offset.
	The store instructions SW, SH and SB work in the same way. SW stands for store word, SH store halfword and SB means store byte. The commands are structured in the same way as the loading commands are. The left side of the command is the register to take the value from and the second parameter is the register, containing the address, and the offset, where the value should be stored.  \cite{Berkeley2019} \\
	ARM instead has a few more instructions but the basics are almost the same. The (LDR) command can be extended by an B to load only a byte, SB to load a signed byte, H to load a half word, SH to load an signed halfword and SW to load a signed word. To store a value, there are 3 possible ways STR to store the complete register, STRB to store a byte and STRH to store a halfword. These basic load and store commands are followed by: load/store pairs, unscaled offsets and much more. Because there is no such possibility in RISC-V, there is no comparison about them. \cite{Arm2020}
	
	To compare the further complexity of the instructions itself, it is important to compare the different instruction formats which are extending the load/store instruction format.
	In RISC-V the RV32I has different instruction formats. The base-instruction formats (R,I,S,U), the immediate encoding variants (B,J), integer computational instructions containing integer register-immediate instructions and integer register-register operations and NOP instructions, control transfer instructions consisting of unconditional jumps and conditional branches, memory model, control and status register instructions (CSR Instructions, Timer and counters) and environment call and breakpoints as well as the store and load instructions, which are more detailed explained above.
	In RISC-V the RV64I and the RV128I contain the same instruction formats but with a few internal difference because of the different register size. \cite{Waterman2017}
	\\
	The ARMv8 has Branches, Exception generating and System instructions, load and stores, data processing -immediate, data processing - register, data processing - SIMD and floating-point instruction formats. Which are counted less than the RISC-V instruction formats but they are containing much more subformats than the once from RISC-V.
	TODO: Here the instruction formats of ARM to cover the once of RISC-V. \cite{Arm2020}
	%Todo: compare maybe the setup of the ARM instructions (are there more different/complex ways to use them??)
	%Write some stuff comparing machine-level and supervisor level \cite{AndrewWaterman} with 3 basic architecture profiles (Application profil, real-time profil, microcontroller profil\cite{ArmManual} S. 36
	%godbolt.org
	%basic comparing between basic
	
	\subsection{Performance (written by Alexander Schmid)}
	Code size is the performance parameter most influenced by the \gls{ISA} and not by the \gls{CPU}'s implementation,
	given that the \gls{ISA} and the compiler are the only two factors that influence code size.
	For the other performance aspects, the influence of the \gls{ISA} is debatable \cite{Blem2013} \cite{Akram2017},
	so code size shall be the main focus of the performance comparison.
	Given that code size is most critical in embedded applications, the Embench benchmark suite is a good benchmark
	with which to compare code sizes. It consists of a number of programs frequently used in embedded
	applications, such as CRC, signal filtering, AES and QR code reading. \cite{Patterson2019}
	When compiling the Embench suite for both RV32IMC as well as 32-bit ARM with the Thumb-2 extension using GCC 7,
	the code for RISC-V is approximately 11\% larger than the code for ARM. \cite{Perotti2020}
	Part of this gap in code size can be explained by the relative immaturity of the RISC-V implementation of
	GCC. RISC-V was introduced in 2017 and the code size of the Embench suite
	compiled for RISC-V is lower with newer versions of GCC, however it is still larger than the code generated for ARM as of 2019. \cite{Patterson2019}

	In \cite{Perotti2020} an extension for RISC-V is introduced, called HCC, that is aimed at reducing the code size of RISC-V.
	This extension brings the code size gap down to 2.2\% for the Embench suite
	and makes the RISC-V code smaller than ARM by 1.75\% in a proprietary IoT benchmark developed by Huawei. \cite{Perotti2020}
	
	TODO: In subsequent submissions, mention RV64 being significantly smaller than AArch64 as described in \cite[page 62]{Waterman2016},
	and possibilities of comparing execution speed and energy efficiency as described in \cite{Blem2013} and \cite{Akram2017}.

	\subsection{Extensibility}\label{sec:extensibility}
	What instruction set extensions are there for both ISAs? Who can develop new extensions?
	\subsection{Ecosystem}
	Which compilers support ARM and RISC-V? Which operating systems and libraries?


\section{Discussion}
\label{ref:discussion}
	\subsection{Advantages of ARM}
	What are the advantages of ARM compared to RISC-V?
	\subsection{Advantages of RISC-V}
	What are the advantages of RISC-V compared to ARM?
	\subsection{Future directions and challenges}
	How can we more accurately measure performance differences between ARM and RISC-V and how do ISA extensions affect performance?

\section{Conclusion and Outlook}
\label{ref:conclusion}
	\subsection{Summary of results}
	\subsection{Interpretation of results}
	\subsection{Future directions}


\section{Overview of Literature}
Alexander Schmid \cite{Akram2017} \cite{Arm2020} \cite{Asanovic2014} \cite{HeuiLee2001} \cite{Patterson2019} \cite{Perotti2020} \cite{Shore2015} \cite{Waterman2016} \cite{Xu2003}

Florian Henneke \cite{Waterman2016} \cite{Ryzhyk2006} \cite{Asanovic2014} \cite{Furber2000} \cite{Microsoft2020} \cite{Greenwaves2020} \cite{Aws2020} \cite{Microsoft2020}

Michael Schneider \cite{Waterman2017} \cite{Arm2020} \cite{George1990} \cite{Waterman2016} \cite{50years} \cite{hennessy2012computer} \cite{drechsler2020enhanced} \cite{WisconsinMadison2016} \cite{IEEE2018} \cite{Dirvin2019} \cite{Bandic2019} \cite{Berkeley2019}

\bibliographystyle{IEEEtran}
\bibliography{bibliography}

\end{document}
